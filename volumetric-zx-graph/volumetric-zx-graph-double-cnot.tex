\documentclass[tikz, preview, border=1pt]{standalone}

\usepackage{pgfplots}
\pgfplotsset{compat=1.18}

\usepackage{ifthen}
\usepackage{xstring}

\usepackage{tikz}

\tikzstyle{quantum red}=[red!75!black!50]
\tikzstyle{quantum green}=[green!75!black!50]
\tikzstyle{quantum blue}=[ blue!75!black!50]
\tikzstyle{boundary}=[darkgray!90]

\tikzstyle{edge}=[line cap = round, ultra thick]
\tikzstyle{identity}=[darkgray!90]
\tikzstyle{hadamard}=[yellow!75!black!75]

\tikzstyle{axis}=[line cap=round, thick]
\tikzstyle{U}=[gray!75]
\tikzstyle{X}=[  red!75!black!50]
\tikzstyle{Y}=[green!75!black!50]
\tikzstyle{Z}=[ blue!75!black!50]

\tikzset{
	% Graph parameters
	zx/graph/labels/.initial=hide,
	zx/graph/labels/.store in=\ShowNodeLabels,
	% Node parameters
	zx/node/type/.initial=R,
	zx/node/type/.store in=\NodeType,
	zx/node/identifier/.store in=\Identifier,
	% Edge parameters
	zx/edge/axis/.store in=\EdgeAxis,
	zx/edge/type/.store in=\EdgeType,
	% Plane parameters
	zx/plane/node/.store in=\Identifier,
	zx/plane/orientation/.store in=\Orientation
}

\newcommand{\Node}[2][]{
	\begingroup
		\tikzset{zx/node/.cd, #1}

		\coordinate (node\Identifier) at #2;

		\IfStrEqCase{\NodeType}{
			{U}{\fill[lightgray] (node\Identifier) circle (3pt);}
			{X}{\fill[X] (node\Identifier) circle (3pt);}
			{Y}{\fill[Y] (node\Identifier) circle (3pt);}
			{Z}{\fill[Z] (node\Identifier) circle (3pt);}
			{O}{\fill[boundary, line width = 1pt] (node\Identifier) circle (2pt);}
		}

		\node (\Identifier) at (node\Identifier) {};
		\ifdefined\ShowNodeLabels
			\IfStrEqCase{\ShowNodeLabels}{
				{show}{\node[white, scale=0.25] at (node\Identifier) {\tt\Identifier};}
				{hide}{}
			}
		\fi

	\endgroup
}

\newcommand{\Edge}[3][]{
	\begingroup
		\tikzset{zx/edge/.cd, #1}
		\ifdefined\EdgeType
			\draw[edge, \EdgeType, shorten <= 6pt, shorten >= 6pt] (node#2) -- (node#3);
		\else
			\draw[edge, identity, shorten <= 6pt, shorten >= 6pt] (node#2) -- (node#3);
		\fi
		\ifdefined\EdgeAxis
			\draw[edge, thick, \EdgeAxis, shorten <= 6pt, shorten >= 6pt] (node#2) -- (node#3);
		\fi
	\endgroup
}

\newcommand{\Plane}[2][]{
	\begingroup
		\tikzset{zx/plane/.cd, #1}

		\IfStrEqCase{\Orientation}{
			{X}{\draw[quantum red, line width = 1pt] (node\Identifier) circle (4pt);}
			{Y}{\draw[quantum green, line width = 1pt] (node\Identifier) circle (4pt);}
			{Z}{\draw[quantum blue, line width = 1pt] (node\Identifier) circle (4pt);}
		}
	\endgroup
}

\newcommand{\ZXGraph}[3][]{
	\begingroup
		\tikzset{zx/graph/.cd, #1}
		\begin{tikzpicture}
			{#2} % Layout of the ZX-Graph (i.e. nodes and edges)
			{#3} % Draw plane-constraints on spiders
		\end{tikzpicture}
	\endgroup
}

%A spider with at most four legs must have all its leg in the same plane (X, Y or Z) \\

%\sc Minimal length of path connecting:
%\begin{itemize}
%	\item Two adjacent spiders of the same color lying in the same plane : 1
%	\item Two adjacent spiders of different colors lying in the same plane : 3
%\end{itemize}

% Assigning plane constraints may increase the Manhattan Distance between two spiders
% This means they are not adjacent in 3D and require a sequence of edges to connect them.

\begin{document}

	% Double-CNOT
	\ZXGraph[labels=show]{
		\Node[type=O, identifier=B1]{(0, 0)}
		\Node[type=O, identifier=B2]{(0,-1)}

		\Node[type=O, identifier=B3]{(3, 0)}
		\Node[type=O, identifier=B4]{(3,-1)}

		\Node[type=X, identifier=X1]{(1,-1)}
		\Node[type=Z, identifier=Z1]{(1, 0)}

		\Node[type=Z, identifier=Z2]{(2,-1)}
		\Node[type=X, identifier=X2]{(2, 0)}

		\Edge{B1}{Z1}
		\Edge{X1}{Z2}
		\Edge{Z2}{B4}

		\Edge{B2}{X1}
		\Edge{Z1}{X2}
		\Edge{X2}{B3}

		\Edge{X1}{Z1}
		\Edge{Z2}{X2}
	}{
		% No plane constraints
	}

	% Double-CNOT (w. initial plane constraints)
	\ZXGraph[labels=show]{
		\Node[type=O, identifier=B1]{(+1,+1)}
		\Node[type=O, identifier=B2]{( 0,-1)}

		\Node[type=O, identifier=B3]{(+2.6,+1.6)}
		\Node[type=O, identifier=B4]{(+3,-1)}

		\Node[type=Z, identifier=Z1]{(+1, 0)}
		\Node[type=X, identifier=X1]{(+1,-1)}

		\Node[type=Z, identifier=Z2]{(+2,-1)}
		\Node[type=X, identifier=X2]{(+1.8,+0.8)}

		\Edge[axis=Z]{B1}{Z1}
		\Edge[axis=Y]{X1}{Z2}
		\Edge[axis=Y]{Z2}{B4}

		\Edge[axis=Y]{B2}{X1}
		\Edge[axis=X]{Z1}{X2}
		\Edge[axis=X]{X2}{B3}

		\Edge[axis=Z]{X1}{Z1}
		\Edge[axis=U]{Z2}{X2}

		% Given the colors and planes assigned to X1, Z1 and Z2, the shortest path between spiders Z2 and X2
		% has a Manhattan Distance of at least 3, which means two extra spiders are needed.
		\Edge[axis=U]{Z2}{X2}
	}{
		\Plane[node=Z1, orientation=Y]{}
		\Plane[node=X1, orientation=X]{}

		\Plane[node=X2, orientation=Z]{}
		\Plane[node=Z2, orientation=Z]{}
	}

	% Double-CNOT (w. initial plane constraints)
	\ZXGraph[labels=show]{
		\Node[type=O, identifier=B1]{(+1,+1)}
		\Node[type=O, identifier=B2]{( 0,-1)}

		\Node[type=O, identifier=B3]{(+2.6,+1.6)}
		\Node[type=O, identifier=B4]{(+3,-1)}

		\Node[type=Z, identifier=Z1]{(+1, 0)}
		\Node[type=X, identifier=X1]{(+1,-1)}

		\Node[type=Z, identifier=Z2]{(+2,-1)}
		\Node[type=X, identifier=X2]{(+1.8,+0.8)}

		\Edge[axis=Z]{B1}{Z1}
		\Edge[axis=Y]{X1}{Z2}
		\Edge[axis=Y]{Z2}{B4}

		\Edge[axis=Y]{B2}{X1}
		\Edge[axis=X]{Z1}{X2}
		\Edge[axis=X]{X2}{B3}

		\Edge[axis=Z]{X1}{Z1}

		% The colours of the two extra spiders (i.e. E1, E2) can be inferred from the plane constraints
		% along with the colours of their respective neighbour.
		\Node[type=X, identifier=E1]{(+2.8,+0.8)}
		\Node[type=Z, identifier=E2]{(+2.8,-0.2)}

		\Edge[axis=Y]{X2}{E1}
		\Edge[axis=Z]{E1}{E2}
		\Edge[axis=X]{E2}{Z2}
	}{
		\Plane[node=Z1, orientation=Y]{}
		\Plane[node=X1, orientation=X]{}

		\Plane[node=X2, orientation=Z]{}
		\Plane[node=Z2, orientation=Z]{}

		\Plane[node=E1, orientation=X]{}
		\Plane[node=E2, orientation=Y]{}
	}

\end{document}