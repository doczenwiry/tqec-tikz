\documentclass[tikz, preview, border=1pt]{standalone}

\usepackage{pgfplots}
\pgfplotsset{compat=1.18}

\usepackage{xstring}
\usepackage{xkeyval}

\usepackage{tikz}

\tikzstyle{quantum red}=[red!75!black!50]
\tikzstyle{quantum green}=[green!75!black!50]
\tikzstyle{quantum blue}=[ blue!75!black!50]
\tikzstyle{boundary}=[darkgray!90]

\tikzstyle{edge}=[line cap = round, ultra thick]
\tikzstyle{identity}=[darkgray!90]
\tikzstyle{hadamard}=[yellow!75!black!75]

\tikzset{
	% Graph parameters
	zx/graph/labels/.initial=hide,
	zx/graph/labels/.store in=\ShowNodeLabels,
	% Node parameters
	zx/node/type/.initial=R,
	zx/node/type/.store in=\NodeType,
	zx/node/identifier/.store in=\Identifier,
	% Edge parameters
	zx/edge/type/.store in=\EdgeType,
	% Plane parameters
	zx/plane/node/.store in=\Identifier,
	zx/plane/orientation/.store in=\Orientation
}

\newcommand{\Node}[2][]{
	\begingroup
		\tikzset{zx/node/.cd, #1}

		\coordinate (node\Identifier) at #2;

		\IfStrEqCase{\NodeType}{
			{R}{\fill[quantum red] (node\Identifier) circle (3pt);}
			{G}{\fill[quantum green] (node\Identifier) circle (3pt);}
			{B}{\fill[quantum blue] (node\Identifier) circle (3pt);}
			{O}{\draw[boundary, line width = 1pt] (node\Identifier) circle (4pt);}
		}

		\node (\Identifier) at (node\Identifier) {};
		\ifdefined\ShowNodeLabels
			\IfStrEqCase{\ShowNodeLabels}{
				{show}{\node[scale=0.25] at (node\Identifier) {\textsc{\Identifier}};}
				{hide}{}
			}
		\fi

	\endgroup
}

\newcommand{\Edge}[3][]{
	\begingroup
		\tikzset{zx/edge/.cd, #1}
		\ifdefined\EdgeType
			\draw[edge, \EdgeType, shorten <= 6pt, shorten >= 6pt] (node#2) -- (node#3);
		\else
			\draw[edge, identity, shorten <= 6pt, shorten >= 6pt] (node#2) -- (node#3);
		\fi
	\endgroup
}

\newcommand{\Plane}[2][]{
	\begingroup
		\tikzset{zx/plane/.cd, #1}

		\IfStrEqCase{\Orientation}{
			{X}{\draw[quantum red, line width = 1pt] (node\Identifier) circle (4pt);}
			{Y}{\draw[quantum green, line width = 1pt] (node\Identifier) circle (4pt);}
			{Z}{\draw[quantum blue, line width = 1pt] (node\Identifier) circle (4pt);}
		}
	\endgroup
}

\newcommand{\ZXGraph}[3][]{
	\begingroup
		\tikzset{zx/graph/.cd, #1}
		\begin{tikzpicture}
			{#2} % Layout of the ZX-Graph (i.e. nodes and edges)
			{#3} % Draw plane-constraints on spiders
		\end{tikzpicture}
	\endgroup
}

%A spider with at most four legs must have all its leg in the same plane (X, Y or Z) \\
%
%\sc Minimal length of path connecting:
%\begin{itemize}
%	\item Two adjacent spiders of the same color lying in the same plane : 1
%	\item Two adjacent spiders of different colors lying in the same plane : 3
%\end{itemize}

\begin{document}
	% CNOT
	\ZXGraph{
		\Node[type=O, identifier=B1]{(-1, 0)}
		\Node[type=R, identifier=X1]{( 0, 0)}
		\Node[type=O, identifier=B2]{(-1,-1)}

		\Node[type=O, identifier=B3]{(+1, 0)}
		\Node[type=B, identifier=Z1]{( 0,-1)}
		\Node[type=O, identifier=B4]{(+1,-1)}

		\Edge{X1}{B1}
		\Edge{X1}{B3}
		\Edge{X1}{Z1}
		\Edge{Z1}{B2}
		\Edge{Z1}{B4}
	}{
		\Plane[node=X1, orientation=X]{}
		\Plane[node=Z1, orientation=Z]{}
	}

	% Double-CNOT
	\ZXGraph{
		\Node[type=O, identifier=B1]{(0, 0)}
		\Node[type=O, identifier=B2]{(0,-1)}

		\Node[type=O, identifier=B3]{(3, 0)}
		\Node[type=O, identifier=B4]{(3,-1)}

		\Node[type=R, identifier=X1]{(1, 0)}
		\Node[type=B, identifier=Z1]{(1,-1)}

		\Node[type=B, identifier=Z2]{(2, 0)}
		\Node[type=R, identifier=X2]{(2,-1)}

		\Edge{B1}{X1}
		\Edge{X1}{Z2}
		\Edge{Z2}{B3}

		\Edge{B2}{Z1}
		\Edge{Z1}{X2}
		\Edge{X2}{B4}

		\Edge{X1}{Z1}
		\Edge{Z2}{X2}
	}{
		\Plane[node=X1, orientation=X]{}
		\Plane[node=Z1, orientation=Y]{}

		\Plane[node=Z2, orientation=Y]{}
		\Plane[node=X2, orientation=X]{}
	}

	% Double-CNOT
	\ZXGraph{
		\Node[type=O, identifier=B1]{(0, 0)}
		\Node[type=O, identifier=B2]{(0,-1)}

		\Node[type=O, identifier=B3]{(3, 0)}
		\Node[type=O, identifier=B4]{(3,-1)}

		\Node[type=R, identifier=X1]{(1, 0)}
		\Node[type=B, identifier=Z1]{(1,-1)}

		\Node[type=B, identifier=Z2]{(2, 0)}
		\Node[type=R, identifier=X2]{(2,-1)}

		\Edge{B1}{X1}
		\Edge{X1}{Z2}
		\Edge{Z2}{B3}

		\Edge{B2}{Z1}
		\Edge{Z1}{X2}
		\Edge{X2}{B4}

		\Edge{X1}{Z1}
		\Edge{Z2}{X2}
	}{
		\Plane[node=X1, orientation=X]{}
		\Plane[node=Z1, orientation=Y]{}

		\Plane[node=Z2, orientation=Z]{}
		\Plane[node=X2, orientation=Z]{}
	}

	% Double-CNOT
	\ZXGraph{
		\Node[type=O, identifier=B1]{(0, 0)}
		\Node[type=O, identifier=B2]{(0,-1)}

		\Node[type=O, identifier=B3]{(3, 0)}
		\Node[type=O, identifier=B4]{(3,-1)}

		\Node[type=R, identifier=X1]{(1, 0)}
		\Node[type=B, identifier=Z1]{(1,-1)}

		\Node[type=B, identifier=Z2]{(2, 0)}
		\Node[type=R, identifier=X2]{(2,-1)}

		\Edge{B1}{X1}
		\Edge{X1}{Z2}
		\Edge{Z2}{B3}

		\Edge{B2}{Z1}
		\Edge{Z1}{X2}
		\Edge{X2}{B4}

		\Edge{X1}{Z1}
		\Edge{Z2}{X2}
	}{
		\Plane[node=X1, orientation=X]{}
		\Plane[node=Z1, orientation=Y]{}

		\Plane[node=Z2, orientation=Y]{}
		\Plane[node=X2, orientation=Z]{}
	}

\end{document}