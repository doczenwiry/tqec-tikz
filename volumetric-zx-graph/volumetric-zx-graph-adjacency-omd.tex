\documentclass[tikz, preview, border=1pt]{standalone}

\usepackage{pgfplots}
\pgfplotsset{compat=1.18}

\usepackage{ifthen}
\usepackage{xstring}

\usepackage{tikz}

\tikzstyle{edge}=[line cap = round, ultra thick]
\tikzstyle{identity}=[darkgray!90]
\tikzstyle{hadamard}=[yellow!75!black!75]

\tikzstyle{axis}=[line cap=round, thick]
\tikzstyle{U}=[lightgray]
\tikzstyle{X}=[  red!75!black!50]
\tikzstyle{Y}=[green!75!black!50]
\tikzstyle{Z}=[ blue!75!black!50]
\tikzstyle{O}=[darkgray!90, line width = 1pt]

\tikzset{
	% Graph parameters
	zx/graph/labels/.initial=hide,
	zx/graph/labels/.store in=\ShowNodeLabels,
	% Node parameters
	zx/node/type/.initial=R,
	zx/node/type/.store in=\NodeType,
	zx/node/identifier/.store in=\Identifier,
	% Edge parameters
	zx/edge/axis/.store in=\EdgeAxis, % Edge represents a change of +1/-1 between two adjacent spiders along an axis
	zx/edge/type/.store in=\EdgeType, % Edge can be identity or hadamard
	% Plane parameters
	zx/plane/node/.store in=\Identifier,
	zx/plane/normal/.store in=\NormalVector
}

\newcommand{\Node}[2][]{
	\begingroup
		\tikzset{zx/node/.cd, #1}

		\coordinate (node\Identifier) at #2;

		\ifthenelse{\equal{\NodeType}{O}}{
			\fill[\NodeType] (node\Identifier) circle (2pt);
		}{
			\fill[\NodeType] (node\Identifier) circle (3pt);
		}

		\node (\Identifier) at (node\Identifier) {};
		\ifdefined\ShowNodeLabels
			\IfStrEqCase{\ShowNodeLabels}{
				{show}{\node[white, scale=0.25] at (node\Identifier) {\tt\Identifier};}
				{hide}{}
			}
		\fi

	\endgroup
}

\newcommand{\Edge}[3][]{
	\begingroup
		\tikzset{zx/edge/.cd, #1}
		\ifdefined\EdgeType
			\draw[edge, \EdgeType, shorten <= 6pt, shorten >= 6pt] (node#2) -- (node#3);
		\else
			\draw[edge, identity, shorten <= 6pt, shorten >= 6pt] (node#2) -- (node#3);
		\fi
		\ifdefined\EdgeAxis
			\draw[edge, thick, \EdgeAxis, shorten <= 6pt, shorten >= 6pt] (node#2) -- (node#3);
		\fi
	\endgroup
}

\newcommand{\Plane}[2][]{
	\begingroup
		\tikzset{zx/plane/.cd, #1}

		\draw[\NormalVector, line width = 1pt] (node\Identifier) circle (4pt);
	\endgroup
}

\newcommand{\ZXGraph}[3][]{
	\begingroup
		\tikzset{zx/graph/.cd, #1}
		\begin{tikzpicture}
			{#2} % Layout of the ZX-Graph (i.e. nodes and edges)
			{#3} % Draw plane-constraints on spiders
		\end{tikzpicture}
	\endgroup
}

% A spider with at most four legs must have all its leg in the same plane (X, Y or Z)

%\sc Minimal length of path connecting:
%\begin{itemize}
%	\item Two adjacent spiders of the same color lying in the same plane : 1
%	\item Two adjacent spiders of different colors lying in the same plane : 3
%\end{itemize}

% Assigning plane constraints may increase the Manhattan Distance between two spiders
% This means they are not adjacent in 3D and require a sequence of edges to connect them.

\begin{document}
	% Two RED spiders with distances (1, 0, 0)
	% Optimal Manhattan Distance = 1; Minimal overhead: none.
	\ZXGraph[labels=show]{
		\Node[type=X, identifier=S1]{( 0, 0)}
		\Node[type=X, identifier=S2]{(+1, 0)}

		\Edge{S1}{S2}
	}{
		% No plane constraints
	}

	% Two RED spiders in same PLANE with distances (1, 0, 0) has Opt. MD=6. Minimal overhead: none.
	\ZXGraph[labels=show]{
		\Node[type=X, identifier=S1]{( 0, 0)}
		\Node[type=X, identifier=S2]{(+1, 0)}

		\Edge[axis=Y]{S1}{S2}
	}{
		\Plane[node=S1, normal=X]{}
		\Plane[node=S2, normal=X]{}
	}

	% Two RED spiders in different PLANE with distances (1, 0, 0)
	% Optimal Manhattan Distance = 7; overhead: +6.
	\ZXGraph[labels=show]{
		\Node[type=X, identifier=S1]{( 0, 0)}
		\Node[type=Z, identifier=S2]{(+1, 0)}

		\Node[type=X, identifier=E1]{( 0.0, +1.0)}
		\Node[type=Z, identifier=E2]{( 0.0, +2.0)}
		\Node[type=X, identifier=E3]{( 0.7, +2.7)}
		\Node[type=Z, identifier=E4]{( 1.7, +2.7)}
		\Node[type=X, identifier=E5]{( 1.7, +1.7)}
		\Node[type=X, identifier=E6]{( 1.0, +1.0)}

		\Edge[axis=Z]{S1}{E1}
		\Edge[axis=Z]{E1}{E2}
		\Edge[axis=X]{E2}{E3}
		\Edge[axis=Y]{E3}{E4}
		\Edge[axis=Z]{E4}{E5}
		\Edge[axis=X]{E5}{E6}
		\Edge[axis=Z]{E6}{S2}

	}{
		\Plane[node=S1, normal=X]{}
		\Plane[node=S2, normal=X]{}

		\Plane[node=E1, normal=X]{}
		\Plane[node=E2, normal=Y]{}
		\Plane[node=E3, normal=Z]{}
		\Plane[node=E4, normal=X]{}
		\Plane[node=E5, normal=Y]{}
		\Plane[node=E6, normal=Y]{}
	}

\end{document}