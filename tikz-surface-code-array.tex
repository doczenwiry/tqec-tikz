\usepackage{pgfplots}
\pgfplotsset{compat=1.18}

\usepackage{ifthen}
\usepackage{xstring}

\usepackage{tikz}
\usetikzlibrary{math}

\tikzstyle{data}=[darkgray!80]
\tikzstyle{stabilizer}=[line width = 1pt, opacity=0.35]
\tikzstyle{X}=[  red!75!black!75]
\tikzstyle{Y}=[green!75!black!75]
\tikzstyle{Z}=[ blue!75!black!75]

\makeatletter
	\define@key{sca:main}{order}{\def\Order{#1}} % Distance = 2*Order + 1
	\define@key{sca:main}{qubits}{\def\ShowQubits{#1}}
	\define@key{sca:main}{labels}{\def\ShowLabels{#1}}
	% Needed to properly place the 2-qubit stabilizers.
	% This can be inferred from the parity of the coordinates but I don't know how to do that in tikz-math.
	\define@key{sca:main}{offset}{\def\Offset{#1}}
	\define@key{sca:main}{coordinates}{\def\ShowCoordinates{#1}}
	% Controls the offset of the 2-qubit stabilizers along the perimeter (0 or 1)
	\define@key{sca:main}{perimeter}{\def\Perimeter{#1}}
	% Controls whether there is a Hadamard flip on the surface code array
	\define@key{sca:main}{hadamard}{\def\Hadamard{#1}}

	\setkeys{sca:main}{qubits=show, labels=hide, perimeter=0, hadamard=0, coordinates = hide}

	\define@key{sca:qubit}{position}{\def\Position{#1}}
	\define@key{sca:qubit}{orderH}{\def\OrderH{#1}}
	\define@key{sca:qubit}{orderW}{\def\OrderW{#1}}

	\define@key{sca:gate}{type}{\def\GateType{#1}} % X or Z
\makeatother

\tikzmath{
	function Plaquette(\RowDistance, \ColDistance, \Row, \Col, \Opacity) {
		% Place two-qubits stabilizers on the LEFT
		if \Col == 0 && Mod(\Row, \RowDistance) != 0 && Mod(\Row+\Col, 2) == \Offset then {
			if Mod(\Row+\Col, 2) == \Hadamard then {
				{ % Z-stabilizers
					\fill[stabilizer, Z, opacity = \Opacity] (\Col + 1 - \Trim, \Row + \Trim) 
					    arc[start angle=270, end angle=90, radius=0.5 - \Trim]
					    -- (\Col + 1 - \Trim, \Row + \Trim) -- cycle;
					\node[Z, opacity = \Opacity] (M\Col\Row) at (\Col + 0.775, \Row + 0.5) {
						\ifthenelse{\equal{\ShowLabels}{show}}{\tt\strut Z}{}
					};
				};
			} else {
				{ % X-stabilizers
					\fill[stabilizer, X, opacity = \Opacity] (\Col + 1 - \Trim, \Row + \Trim) 
					    arc[start angle=270, end angle=90, radius=0.5 - \Trim]
					    -- (\Col + 1 - \Trim, \Row + \Trim) -- cycle;
					\node[X, opacity = \Opacity] (M\Col\Row) at (\Col + 0.775, \Row + 0.5) {
						\ifthenelse{\equal{\ShowLabels}{show}}{\tt\strut X}{}
					};
				};
			};
		};
		% Place two-qubits stabilizers on the RIGHT
		if \Col == \ColDistance && Mod(\Row, \RowDistance) != 0 && Mod(\Row+\Col, 2) == \Offset then {
			if Mod(\Row+\Col, 2) == \Hadamard then {
				{ % Z-stabilizers
					\fill[stabilizer, Z, opacity = \Opacity] (\Col + \Trim, \Row + \Trim) 
					    arc[start angle=270, end angle=450, radius=0.5 - \Trim]
					    -- (\Col + \Trim, \Row + \Trim) -- cycle;
					\node[Z, opacity = \Opacity] (M\Col\Row) at (\Col + 0.225, \Row + 0.5) {
						\ifthenelse{\equal{\ShowLabels}{show}}{\tt\strut Z}{}
					};
				};
			} else {
				{ % X-stabilizers
					\fill[stabilizer, X, opacity = \Opacity] (\Col + \Trim, \Row + \Trim) 
					    arc[start angle=270, end angle=450, radius=0.5 - \Trim]
					    -- (\Col + \Trim, \Row + \Trim) -- cycle;
					\node[X, opacity = \Opacity] (M\Col\Row) at (\Col + 0.225, \Row + 0.5) {
						\ifthenelse{\equal{\ShowLabels}{show}}{\tt\strut X}{}
					};
				};
			};
		};
		% Place four-qubits stabilizers WITHIN the array
		if Mod(\Col, \ColDistance) != 0 && Mod(\Row, \RowDistance) != 0 then {
			if Mod(\Row+\Col, 2) == \Hadamard then {
				{
					\fill[stabilizer, Z, opacity = \Opacity] (\Col + \Trim, \Row + \Trim)
						rectangle (\Col + 1 - \Trim, \Row + 1 - \Trim);
					\node[Z, opacity = \Opacity] (M\Col\Row) at (\Col + 0.5, \Row + 0.5) {
						\ifthenelse{\equal{\ShowLabels}{show}}{\tt\strut Z}{}
					};
				};
			} else {
				{
					\fill[stabilizer, X, opacity = \Opacity] (\Col + \Trim, \Row + \Trim)
						rectangle (\Col + 1 - \Trim, \Row + 1 - \Trim);
					\node[X, opacity = \Opacity] (M\Col\Row) at (\Col + 0.5, \Row + 0.5) {
						\ifthenelse{\equal{\ShowLabels}{show}}{\tt\strut X}{}
					};
				};
			};
		};
		% Place two-qubits X-stabilizers at the BOTTOM
		if \Row == 0 && Mod(\Col, \ColDistance) != 0 && Mod(\Row+\Col, 2) != \Offset then {
			if Mod(\Row+\Col, 2) == \Hadamard then {
				{
					\fill[stabilizer, Z, opacity = \Opacity] (\Col + \Trim,\Row + 1 - \Trim) 
					    arc[start angle=180, end angle=360, radius=0.5 - \Trim]
					    -- (\Col + 1 - \Trim, \Row + 1 - \Trim) -- cycle;
					\node[Z, opacity = \Opacity] (M\Col\Row) at (\Col + 0.5, \Row + 0.725) {
						\ifthenelse{\equal{\ShowLabels}{show}}{\tt\strut Z}{}
					};
				};
			} else {
				{
					\fill[stabilizer, X, opacity = \Opacity] (\Col + \Trim,\Row + 1 - \Trim) 
					    arc[start angle=180, end angle=360, radius=0.5 - \Trim]
					    -- (\Col + 1 - \Trim, \Row + 1 - \Trim) -- cycle;
					\node[X, opacity = \Opacity] (M\Col\Row) at (\Col + 0.5, \Row + 0.725) {
						\ifthenelse{\equal{\ShowLabels}{show}}{\tt\strut X}{}
					};
				};
			};
		};
		% Place two-qubits X-stabilizers at the TOP
		if \Row == \RowDistance && Mod(\Col, \ColDistance) != 0 && Mod(\Row+\Col, 2) != \Offset then {
			if Mod(\Row+\Col, 2) == \Hadamard then {
				{
					\fill[stabilizer, Z, opacity = \Opacity] (\Col + \Trim,\Row + \Trim) 
					    arc[start angle=180, end angle=0, radius=0.5 - \Trim]
					    -- (\Col + 1 - \Trim, \Row + \Trim) -- cycle;
					\node[Z, opacity = \Opacity] (M\Col\Row) at (\Col + 0.5, \Row + 0.225) {
						\ifthenelse{\equal{\ShowLabels}{show}}{\tt\strut Z}{}
					};
				};
			} else {
				{
					\fill[stabilizer, X, opacity = \Opacity] (\Col + \Trim,\Row + \Trim) 
					    arc[start angle=180, end angle=0, radius=0.5 - \Trim]
					    -- (\Col + 1 - \Trim, \Row + \Trim) -- cycle;
					\node[X, opacity = \Opacity] (M\Col\Row) at (\Col + 0.5, \Row + 0.225) {
						\ifthenelse{\equal{\ShowLabels}{show}}{\tt\strut X}{}
					};
				};
			};
		};
	};
}

\newcommand{\SurfaceCodeArray}[3][]{
	\begingroup
		\setkeys{sca:main}{#1}
		
		\def\Offset{\Perimeter}

		\begin{tikzpicture}
			\tikzmath{
				real \Trim;
				\Trim = 1 / 128;
				integer \Distance;
				\Distance = 2*\Order + 1;
				integer \h;
				integer \v;
				% Place Stabilizer qubits
				for \row in {0,...,\Distance} {
					for \col in {0,...,\Distance} {
						Plaquette(\Distance, \Distance, \row, \col, 0.125);
					};
				};
			}

			% Draw the logical gates beneath the qubits
			{#3}

			\tikzmath{
				integer \row, \col;
				% Place Data qubits
				for \row in {1,...,\Distance} {
					for \col in {1,...,\Distance} {
						{
							\ifthenelse{\equal{\ShowQubits}{show}}{
								\fill[white] (\col, \row) circle (2.5pt);
								\fill[darkgray!25] (\col, \row) circle (2pt);
							}{}
							\ifthenelse{\equal{\ShowCoordinates}{show}}{
								\node at (\col, \row) {\tiny\bf\tt (\col,\row)};
							}{}
						};
					};
				};
			}

			{#2}
		\end{tikzpicture}
	\endgroup
}

\newcommand{\LogicalQubit}[2][]{
	\begingroup
		\setkeys{sca:main,sca:qubit}{#1}

		\ifdefined\OrderH\else\def\OrderH{\Order}\fi
		\ifdefined\OrderW\else\def\OrderW{\Order}\fi

		\def\Hadamard{\Offset} % Should be mod(Offset+Perimeter, 2)

		\begin{scope}[shift={(\Position)}]
			\tikzmath{
				real \Trim;
				\Trim = 1 / 128;
				integer \DistanceH, \DistanceW;
				\DistanceH = 2*\OrderH + 1;
				\DistanceW = 2*\OrderW + 1;
				% Place Stabilizer Plaquettes
				for \row in {0,...,\DistanceH} {
					for \col in {0,...,\DistanceW} {
						Plaquette(\DistanceH, \DistanceW, \row, \col, 0.4);
					};
				};
				% Place Data Qubits
				for \row in {1,...,\DistanceH} {
					for \col in {1,...,\DistanceW} {
						{
							\ifthenelse{\equal{\ShowQubits}{show}}{
								\fill[white] (\col, \row) circle (2.5pt);
								\fill[data] (\col,\row) circle (2pt);
							}{}
						};
					};
				};
			}
		\end{scope}
	\endgroup
}

\newcommand{\LogicalGate}[3][]{
	\begingroup
		\setkeys{sca:gate}{#1}

		\def\Source{#2}
		\def\Target{#3}

		\draw[\GateType, line width = 7pt, line cap = round] \Source -- \Target;
	\endgroup
}